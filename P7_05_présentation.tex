\documentclass[8pt,aspectratio=169,hyperref={unicode=true}]{beamer}

\usefonttheme{serif}
\usepackage{fontspec}
	\setmainfont{TeX Gyre Heros}
\usepackage{unicode-math}
\usepackage{lualatex-math}
	\setmathfont{TeX Gyre Termes Math}
\usepackage{polyglossia}
\setdefaultlanguage[frenchpart=false]{french}
\setotherlanguage{english}
%\usepackage{microtype}
\usepackage[locale = FR,
            separate-uncertainty,
            multi-part-units = single,
            range-units = single]{siunitx}
	\DeclareSIUnit\an{an}
  \DeclareSIUnit{\octet}{o}
\usepackage{amsmath}
\usepackage{amsfonts}
\usepackage{amssymb}
\usepackage{array}
\usepackage{graphicx}
\graphicspath{{./Figures/}}
\usepackage{booktabs}
\usepackage{tabularx}
\usepackage{multirow}
\usepackage{multicol}
    \newcolumntype{L}{>{\raggedright\arraybackslash}X}
    \newcolumntype{R}{>{\raggedleft\arraybackslash}X}
\usepackage{makecell}
\setcellgapes{5pt}
\usepackage{xcolor}
\usepackage{tikz}
\usetikzlibrary{graphs, graphdrawing, arrows.meta} \usegdlibrary{layered, trees}
\usetikzlibrary{overlay-beamer-styles}
\usepackage{subcaption}
\usepackage[]{animate}
\usepackage{float}
\usepackage{csquotes}
\usepackage{minted}

\usetheme[secheader]{Boadilla}
\usecolortheme{seagull}
\setbeamertemplate{enumerate items}[default]
\setbeamertemplate{itemize items}{-}
\setbeamertemplate{navigation symbols}{}
\setbeamertemplate{bibliography item}{}
\setbeamerfont{framesubtitle}{size=\large}
\setbeamertemplate{section in toc}[sections numbered]
%\setbeamertemplate{subsection in toc}[subsections numbered]

\title[Implémentez un modèle de scoring]{Projet 7 : Implémentez un modèle de scoring}
\author[Lancelot \textsc{Leclercq}]{Lancelot \textsc{Leclercq}} 
\institute[]{}
\date[]{\small{5 mai 2022}}

\AtBeginSection[]{
  \begin{frame}
  \vfill
  \centering
    \usebeamerfont{title}\insertsectionhead\par%
  \vfill
  \end{frame}
}

\begin{document}
\setbeamercolor{background canvas}{bg=gray!20}
\begin{frame}[plain]
    \titlepage
\end{frame}

\begin{frame}{Sommaire}
    \Large
    \begin{columns}
        \begin{column}{.7\textwidth}
            \tableofcontents[hideallsubsections]
        \end{column}
    \end{columns}
\end{frame}

\section{Introduction}
\subsection{Problématique}
\begin{frame}{\insertsubsection}
    \begin{columns}
        \begin{column}{.8\textwidth}
            \begin{itemize}
                \item L'entreprise Prêt à dépenser est une société financière qui propose des crédits à la consommation pour des personnes ayant peu ou pas du tout d'historique de prêt
            \end{itemize}
        \end{column}
        \begin{column}{.2\textwidth}
            \includegraphics[width=\textwidth]{./logoPAD.png}
        \end{column}
    \end{columns}
    \begin{itemize}
        \item Objectifs
              \begin{itemize}
                  \item Mettre en œuvre un outil de “scoring crédit” pour calculer la probabilité qu’un client rembourse son crédit
                  \item[]
                  \item Classifier la demande en crédit accordé ou refusé
                        \begin{itemize}
                            \item développer un algorithme de classification en s’appuyant sur des sources de données variées (données comportementales, données provenant d'autres institutions financières, etc.)
                        \end{itemize}
                  \item[]
                  \item Développer un dashboard interactif
                        \begin{itemize}
                            \item expliquer de façon la plus transparente possible les décisions d’octroi de crédit,
                            \item permettre aux clients de disposer de leurs informations personnelles et de les explorer facilement
                        \end{itemize}
              \end{itemize}
    \end{itemize}
\end{frame}

\subsection{Données}
\begin{frame}{\insertsubsection}
    \begin{columns}
        \begin{column}{.4\textwidth}
            \begin{itemize}
                \item Principal fichier utilisé application\_\{train\|test\}.csv
            \end{itemize}
        \end{column}
        \begin{column}{.6\textwidth}
            \includegraphics[width=\textwidth]{./home_credit.png}
        \end{column}
    \end{columns}
\end{frame}

\section{Analyse et traitement des données}
\subsection{Exploration du jeu de données}
\begin{frame}{\insertsection}{\insertsubsection}
    \begin{columns}
        \begin{column}{.4\textwidth}
            \begin{itemize}
                \item Certaines colonnes comportent un grand nombre de données manquantes
                      \begin{itemize}
                          \item Nous utiliserons des modèles résistants à ces données manquantes comme XGBoost et LightGBM
                      \end{itemize}
                \item[]
                \item Encodage des variables catégorielles
                      \begin{itemize}
                          \item par LabelEncoder pour les variables ayant 2 catégories
                          \item par pandas.get\_dummies() pour les variables ayant plus de 2 catégories
                      \end{itemize}
            \end{itemize}
        \end{column}
        \begin{column}{.6\textwidth}
            \includegraphics[width=\textwidth]{app_trainNbDataMiss.pdf}
        \end{column}
    \end{columns}
\end{frame}

\begin{frame}{\insertsection}{\insertsubsection}
    \begin{columns}
        \begin{column}{.5\textwidth}
            \includegraphics[width=\textwidth]{HistPayement.pdf}
        \end{column}
        \begin{column}{.5\textwidth}
            \includegraphics[width=\textwidth]{HistDefAge.pdf}
        \end{column}
    \end{columns}
    \begin{itemize}
        \item L'age des clients semble avoir un impact sur le fait que le client fasse défaut ou non
    \end{itemize}
\end{frame}

\begin{frame}{\insertsection}{\insertsubsection}
    \begin{columns}
        \begin{column}{.33\textwidth}
            \includegraphics[width=\textwidth]{HistSource1.pdf}
        \end{column}
        \begin{column}{.33\textwidth}
            \includegraphics[width=\textwidth]{HistSource2.pdf}
        \end{column}
        \begin{column}{.33\textwidth}
            \includegraphics[width=\textwidth]{HistSource3.pdf}
        \end{column}
    \end{columns}
    \begin{itemize}
        \item Les données EXT\_SOURCE semblent aussi avoir une certaine corrélation avec le fait que le client fasse défaut
    \end{itemize}
\end{frame}

\begin{frame}{\insertsection}{\insertsubsection}
    \begin{columns}
        \begin{column}{.5\textwidth}
            \begin{itemize}
                \item Données sont déséquilibrées : clients faisant défauts peu nombreux par rapport à ceux ne faisant pas défaut
                \item[]
                \item Classer tous les clients comme ne faisant pas défaut \\$\Longrightarrow$ score honorable avec seulement 8\% d'erreurs
                \item[]
                \item Utilisation de la librairie imblearn $\Longrightarrow$ rééchantillonnage de  notre jeu de données
            \end{itemize}
        \end{column}
        \begin{column}{.5\textwidth}
            \includegraphics[width=\textwidth]{DéséquilibreCible.pdf}
        \end{column}
    \end{columns}
\end{frame}

\subsection{Rééchantillonnage du jeux de données}
\begin{frame}{\insertsection}{\insertsubsection}
    Réalisation d'une régression logistique pour essayer les différentes méthodes de rééchantillonnage
    \begin{columns}
        \begin{column}{.5\textwidth}
            \begin{itemize}
                \item Méthodes de sous-échantillonnages :
                      \begin{itemize}
                          \item On réduit le nombre de clients ne faisant pas défaut pour en avoir autant que de client faisant défaut
                          \item[]
                          \item RandomUnderSampler choisi ces derniers au hasard
                          \item[]
                          \item TomekLinks conserve un certain nombre de clients par groupe de clients similaire (repose sur les KNN)
                          \item[]
                      \end{itemize}
            \end{itemize}
        \end{column}
        \begin{column}{.5\textwidth}
            \begin{itemize}
                \item Méthodes de sur-échantillonnages :
                      \begin{itemize}
                          \item On multiplie le nombre de clients faisant défaut pour en avoir autant que des clients ne faisant pas défaut
                          \item[]
                          \item RandomOverSampler dédouble des clients faisant défaut au hasard
                          \item[]
                          \item SMOTE créé de nouveaux clients à partir de groupe de clients similaires
                          \item[]
                      \end{itemize}
            \end{itemize}
        \end{column}
    \end{columns}
    \begin{columns}
        \begin{column}{.5\textwidth}
            \begin{itemize}
                \item Méthodes combinant le sur- et le sous-échantillonnage
                      \begin{itemize}
                          \item SMOTEENN
                          \item SMOTETomek
                      \end{itemize}
            \end{itemize}
        \end{column}
    \end{columns}
\end{frame}

\begin{frame}{\insertsection}{\insertsubsection}
    \vspace{2pt}
    \begin{columns}
        \begin{column}{.25\textwidth}
            {\makegapedcells
                \begin{tabular}{cc|cc}
                    \multicolumn{2}{c}{}
                     & \multicolumn{2}{c}{Prédit}           \\
                     &                            & 0  & 1  \\
                    \cline{2-4}
                    \multirow{2}{*}{\rotatebox[origin=c]{90}{Réel}}
                     & 0                          & TN & FP \\
                     & 1                          & FN & TP \\
                    \cline{2-4}
                \end{tabular}}

            \vspace{15pt}

            \includegraphics[width=\textwidth]{CMNone.pdf}
        \end{column}
        \begin{column}{.25\textwidth}
            \includegraphics[width=\textwidth]{CMRandomUnderSampler.pdf}
            \includegraphics[width=\textwidth]{CMTomekLinks.pdf}
        \end{column}
        \begin{column}{.25\textwidth}
            \includegraphics[width=\textwidth]{CMRandomOverSampler.pdf}
            \includegraphics[width=\textwidth]{CMSMOTE.pdf}
        \end{column}
        \begin{column}{.25\textwidth}
            \includegraphics[width=\textwidth]{CMSMOTETomek.pdf}
            \includegraphics[width=\textwidth]{CMSMOTEENN.pdf}
        \end{column}
    \end{columns}
    \small
    \begin{itemize}
        \item TomekLinks n'est pas plus efficace que sans rééchantillonnage
        \item SMOTEENN fait un grand nombre de faux positifs
        \item SMOTEENN a le plus grand nombre de vrais positifs (TP)
    \end{itemize}
\end{frame}

\section{Optimisation du modèle}
\subsection{GridSearch}
\begin{frame}{\insertsection}{\insertsubsection}
    \begin{itemize}
        \item[] Pour chaque solution de rééquilibrage :
            \begin{itemize}
                \item[] RandomOverSampler, SMOTE, RandomUnderSampler, TomekLinks, SMOTEENN, SMOTETomek
                \item[] \LARGE $\Downarrow$
            \end{itemize}
        \item[] GridSearch pour les deux modèles :
            \begin{itemize}
                \item[] XGBoost Classifier, LightGBM Classifier
                \item[] \LARGE $\Downarrow$
            \end{itemize}
        \item[] Nous retenons le meilleur résultat pour un couple : solution de rééquilibrage/modèle
        \item[]
        \item[] Nous avons fait différents essais d'optimisation
            \begin{itemize}
                \item 1\ier{} essai avec optimisation de l'AUC
                \item 2\nd{} essai avec optimisation de la métrique métier (F\textsubscript{β}-score)
            \end{itemize}
    \end{itemize}
\end{frame}

\subsection{Courbe ROC}
\begin{frame}{\insertsection}{\insertsubsection}
    \begin{columns}
        \begin{column}{.5\textwidth}
            \begin{itemize}
                \item La courbe ROC représente les vrais positifs en fonction des faux positifs
                \item[]
                \item Plus la courbe est proche du coin supérieur gauche meilleur est le modèle
                \item[]
                \item L'aire sous la courbe (AUC) nous donne une valeur numérique pour comparer ces modèles
            \end{itemize}
        \end{column}
        \begin{column}{.5\textwidth}
            \includegraphics[width=\textwidth]{CurvesROCGridBest_base.pdf}
        \end{column}
    \end{columns}
\end{frame}

\subsection{Différentes métriques utilisées}
\begin{frame}{\insertsection}{\insertsubsection}
    \begin{columns}
        \begin{column}{.5\textwidth}
            \begin{itemize}
                \item Accuracy : précision de la classification (somme des éléments bien classé sur le nombre total d'éléments)
                \item AUC : Area Under the Curve, aire sous la courbe ROC
                \item Precision : part de vrais positifs dans les prédictions positives
                \item Recall : part de vrais positifs dans les éléments réellement positifs
                \item F1 : moyenne harmonique de la précision et du rappel
                \item F2 : idem F1 avec un facteur β=2, permettant de mettre plus de poids sur le rappel
            \end{itemize}
        \end{column}
        \begin{column}{.5\textwidth}
            \includegraphics[width=\textwidth]{ScoresGrid_base.pdf}
        \end{column}
    \end{columns}
\end{frame}

\subsection{Création de variables polynomiales}
\begin{frame}{\insertsection}{\insertsubsection}
    \vspace{2pt}
    \begin{columns}
        \begin{column}{.5\textwidth}
            \begin{itemize}
                \item Afin d'améliorer les scores des modèles nous avons essayé de créer des variables polynomiales à partir des colonnes les plus corrélées avec la cible
                \item[]
                \item L'amélioration n'est pas pertinente nous n'avons donc pas conservé ces variables pour notre modèle final
            \end{itemize}
        \end{column}
        \begin{column}{.5\textwidth}
            \includegraphics[width=\textwidth]{ScoresGridFull.pdf}
        \end{column}
    \end{columns}
\end{frame}

\subsection{Métrique métier}
\begin{frame}{\insertsection}{\insertsubsection}
    \begin{columns}
        \begin{column}{.7\textwidth}
            \begin{itemize}
                \item But
                      \begin{itemize}
                          \item Diminuer le nombre de faux négatifs (prédit 0, réel 1) afin d'éviter de manquer des clients qui pourraient potentiellement faire défaut
                          \item[$\longrightarrow$] Améliorer le recall
                      \end{itemize}
            \end{itemize}
        \end{column}
        \begin{column}{.3\textwidth}
            \begin{tikzpicture}
                \node[anchor=south west,inner sep=0] (image) at (0,0) { {\makegapedcells
                            \begin{tabular}{cc|cc}
                                \multicolumn{2}{c}{}
                                 & \multicolumn{2}{c}{Prédit}           \\
                                 &                            & 0  & 1  \\
                                \cline{2-4}
                                \multirow{2}{*}{\rotatebox[origin=c]{90}{Réel}}
                                 & 0                          & TN & FP \\
                                 & 1                          & FN & TP \\
                                \cline{2-4}
                            \end{tabular}}};
                \begin{scope}[x={(image.south east)},y={(image.north west)}]
                    \draw[red, ultra thick] (0.7,0.1) circle [x radius=.9cm, y radius=.3cm];
                    \node[rectangle,below] at (0.7,-0.05) (R) {Recall};
                    \draw[green, ultra thick] (0.87,0.25) circle [x radius=.3cm, y radius=.7cm];
                    \node[rectangle,right] at (1,0.25) (P) {Precision};
                    %\draw[help lines,xstep=.1,ystep=.1] (0,0) grid (1,1);
                    %\foreach \x in {0,1,...,9} { \node [anchor=north] at (\x/10,0) {0.\x}; }
                    %\foreach \y in {0,1,...,9} { \node [anchor=east] at (0,\y/10) {0.\y}; }

                \end{scope}
            \end{tikzpicture}
        \end{column}
    \end{columns}
    \begin{columns}
        \begin{column}{.61\textwidth}
            \begin{itemize}
                \item Outil
                      \begin{itemize}
                          \item Utilisation du $F_β$-score qui permet d'ajouter du poids respectivement au rappel lorsque le facteur β est \num{>1} ou à la précision lorsque le facteur β est \num{<1}
                          \item Utilisation de β=2
                      \end{itemize}
            \end{itemize}
        \end{column}
        \begin{column}{.39\textwidth}
            \includegraphics[width=\textwidth]{ScoresGridF2_base.pdf}
        \end{column}
    \end{columns}
\end{frame}

\section{Explication du modèle}
\subsection{Importance des variables}
\begin{frame}{\insertsection}{\insertsubsection}
    \begin{columns}
        \begin{column}{.5\textwidth}
            \begin{itemize}
                \item Les modèles nous rendent comptent de la part des variables dans leurs résultats
                \item[]
                \item[] Les variables EXT\_SOURCE se retrouve dans les deux cas par exemple
            \end{itemize}
        \end{column}
        \begin{column}{.5\textwidth}
            \includegraphics[width=\textwidth]{BestFeatGrid_base.pdf}
        \end{column}
    \end{columns}
\end{frame}


\subsection{Utilisation de SHAP}
\begin{frame}{\insertsection}{\insertsubsection}
    \begin{itemize}
        \item La librairie SHAP permet d'expliquer le fonctionnement du modèle de manière plus poussée.
        \item Exemples sur le modèle LightGBM :
    \end{itemize}
    \begin{columns}[t]
        \begin{column}{.5\textwidth}
            Fonctionnement global du modèle $\Longrightarrow$ part des variables utilisées lors du classement en général
            \includegraphics[width=.9\textwidth]{shapSummary.pdf}
        \end{column}
        \begin{column}{.5\textwidth}
            Fonctionnement local du modèle $\Longrightarrow$ part des variables utilisées lors du classement d'un élément particulier
            \includegraphics[width=\textwidth]{shapForce.pdf}
            \includegraphics[width=.8\textwidth]{shapWaterfall.pdf}
        \end{column}
    \end{columns}
\end{frame}


\section{Déploiement sur le cloud}
\subsection{API}
\begin{frame}[fragile]{\insertsection}{\insertsubsection}
    \begin{itemize}
        \item Utilisation de Flask
        \item Entrainement du modèle et prédictions
        \item Une URL pour chaque type de données avec une ou plusieurs clés permettant des requêtes sur des clients ou des valeurs particulières
        \item Exemple pour les données des clients pour lesquelles on renseigne l'id du client : \url{http://localhost:5000/ID_clients/infos_client?id=<identifiant>}
        \item[]
    \end{itemize}
    \begin{minted}{python}
@app.route("/ID_clients/infos_client/", methods=["GET"])
def show_data():
    ID_client = request.args.get("id", default=100001, type=int)
    data_client = app_test[app_test.SK_ID_CURR == int(ID_client)].set_index('SK_ID_CURR')
    data_reponse = json.loads(data_client.to_json(orient='index'))
    return jsonify(data_reponse)    
    \end{minted}
\end{frame}

\subsection{Dashboard}
\begin{frame}[fragile]{\insertsection}{\insertsubsection}
    \begin{itemize}
        \item Utilisation de Dashboard
        \item Requêtes à l'API afin d'obtenir les données au format JSON
        \item Exemple pour une requête concernant les données d'un client :
        \item[]
    \end{itemize}
    \begin{minted}{python}
URL_API = 'http://localhost:5000/'

@app.callback(Output('infos_client', 'data'), Input('ID_choosed', 'value'))
def get_client_infos(idclient):
    url = URL_API + 'ID_clients/infos_client/?id=' + str(idclient)
    data = requests.get(url).json()
    return data 
            \end{minted}
\end{frame}

\subsection{Déploiement}
\begin{frame}{\insertsection}{\insertsubsection}
    \begin{columns}
        \begin{column}{.4\textwidth}
            \begin{itemize}
                \item Déploiement sur Heroku
                      \begin{itemize}
                          \item \texttt{heroku git:remote -a bank-scoring-dash}
                          \item \texttt{git add .}
                          \item \texttt{git commit -am 'launch in the cloud'}
                          \item \texttt{git push heroku master}
                      \end{itemize}
            \end{itemize}
        \end{column}
        \begin{column}{.6\textwidth}
            \includegraphics[width=\textwidth]{DashSim.png}
        \end{column}
    \end{columns}
\end{frame}

% \section{Conclusion}
% \begin{frame}{\insertsection}
%     \begin{itemize}
%         \item Utilisation de données déséquilibrées
%         \item[]
%         \item Optimisation des modèles à l'aide d'une GridSearch
%         \item[]
%         \item Utilisation d'une métrique plus particulière afin d'optimiser le modèle en fonction des attentes du métier
%         \item[]
%         \item Il aurait pu être intéressant d'optimiser plus d'hyperparamètres afin d'améliorer un peu plus nos résultats
%         \item[]
%         \item Le déploiement dans le cloud permet un accès rapide à ces données concernant les clients et les prédictions qu'ils fassent ou non défaut
%         \item[]
%         \item Ces données sont accompagnées de graphiques permettant de comparer ces clients avec d'autres clients similaires ou à l'ensemble des clients
%     \end{itemize}
% \end{frame}

\end{document}